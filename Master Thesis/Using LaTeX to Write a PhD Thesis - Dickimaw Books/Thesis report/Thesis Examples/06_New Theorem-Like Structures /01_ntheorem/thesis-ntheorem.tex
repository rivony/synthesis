% The following 2 lines are provided in case you
% want to build your document using arara:
% arara: pdflatex: { synctex: on }
% arara: pdflatex: { synctex: on }
\documentclass[oneside,12pt]{scrbook}

\usepackage{lipsum}

\usepackage{scrhack}

\usepackage[standard]{ntheorem}
\usepackage{setspace}
\usepackage{scrpage2}
\usepackage{listings}

% set up document fonts
\usepackage[T1]{fontenc}
\usepackage{lmodern}

\pagestyle{scrheadings}
\ihead{}
\chead{}
\ohead[]{\headmark}
\ifoot[123456789]{123456789}% registration number
\cfoot[]{}
\ofoot[\pagemark]{\pagemark}

\renewcommand*{\headfont}{\normalfont \itshape}
\renewcommand*{\pnumfont}{\normalfont \bfseries}

\begin{document}

\begin{titlepage}
\centering
\vspace*{1in}
\begin{Large}\bfseries
A Sample PhD Thesis\par
\end{Large}
\vspace{1.5in}
\begin{large}\bfseries
A. N. Other\par
\end{large}
\vfill
A Thesis submitted for the degree of Doctor of Philosophy
\par
\vspace{0.5in}
School of Something
\par
University of Somewhere
\par
\vspace{0.5in}
July 2012
\par
\end{titlepage}

\doublespacing

\frontmatter
\tableofcontents
\listoffigures
\listoftables
\lstlistoflistings

\chapter{Acknowledgements}

I would like to thank my supervisor, Professor Someone. This
research was funded by the Imaginary Research Council.

\chapter{Abstract}

A brief summary of the project goes here.

\mainmatter

\chapter{Introduction}
\label{ch:intro}

\lipsum

\chapter{Technical Introduction}
\label{ch:techintro}

Some sample code is shown in Listing~\ref{lst:sample}.

\lstset{language=C,basicstyle={\ttfamily\singlespacing}}
\begin{lstlisting}[mathescape=true,caption={Sample},label={lst:sample}]
#include <stdio.h> /* needed for printf */
#include <math.h> /* needed for sqrt */

int main()
{
   double x = sqrt(2.0); /* $x = \sqrt{2}$ */

   printf("x = %f\n", x);

   return 1;
}
\end{lstlisting} 

\begin{Definition}[Tautology]\label{def:tautology}
A \emph{tautology} is a proposition that is always true for any
value of its variables.
\end{Definition}

\begin{Definition}[Contradiction]\label{def:contradiction}
A \emph{contradiction} is a proposition that is always false for any
value of its variables.
\end{Definition}

\begin{Theorem}
If proposition $P$ is a tautology
then $\sim P$ is a contradiction,
and conversely.
\begin{Proof}
If $P$ is a tautology, then all elements of its truth table are
true (by Definition~\ref{def:tautology}), so all elements of the truth table for $\sim P$ are false,
therefore $\sim P$ is a contradiction (by
Definition~\ref{def:contradiction}).
\end{Proof}
\end{Theorem}

\begin{Example}\label{ex:rain}
``It is raining or it is not raining'' is a tautology,
but ``it is not raining and it is raining'' is a contradiction.
\end{Example}

\begin{Remark}
Example~\ref{ex:rain} used De Morgan’s Law
$\sim (p \vee q) \equiv \sim p \wedge \sim q$.
\end{Remark}

\lipsum

\chapter{Method}
\label{ch:method}

\lipsum

\chapter{Results}
\label{ch:results}

\lipsum

\chapter{Conclusions}
\label{ch:conc}

\lipsum

\backmatter

% A glossary and list of acronyms may go here
% or may go in the front matter after the abstract.

% The bibliography will go here

\end{document}
