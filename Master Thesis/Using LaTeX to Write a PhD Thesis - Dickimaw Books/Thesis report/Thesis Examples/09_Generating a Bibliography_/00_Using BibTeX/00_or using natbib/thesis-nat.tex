% The following lines are provided in case you want
% to build this document using arara:
% arara: pdflatex: { synctex: on }
% arara: bibtex
% arara: pdflatex: { synctex: on }
% arara: pdflatex: { synctex: on }
\documentclass[oneside,12pt]{scrbook}

\usepackage{scrhack}

\usepackage{amsmath}
\usepackage[standard]{ntheorem}
\usepackage[algo2e]{algorithm2e}
\usepackage{siunitx}
\usepackage[round]{natbib}

\theoremstyle{break}
\theorembodyfont{\normalfont}
\newtheorem{algorithm}[algocf]{Algorithm}

\usepackage{setspace}
\usepackage{scrpage2}
\usepackage{listings}

\pagestyle{scrheadings}
\ihead{}
\chead{}
\ohead[]{\headmark}
\ifoot[123456789]{123456789}% registration number
\cfoot[]{}
\ofoot[\pagemark]{\pagemark}

\renewcommand*{\headfont}{\normalfont \itshape}
\renewcommand*{\pnumfont}{\normalfont \bfseries}

\begin{document}

\begin{titlepage}
\centering
\vspace*{1in}
\begin{Large}\bfseries
A Sample PhD Thesis\par
\end{Large}
\vspace{1.5in}
\begin{large}\bfseries
A. N. Other\par
\end{large}
\vfill
A Thesis submitted for the degree of Doctor of Philosophy
\par
\vspace{0.5in}
School of Something
\par
University of Somewhere
\par
\vspace{0.5in}
July 2012
\par
\end{titlepage}

\doublespacing

\frontmatter
\tableofcontents
\listoffigures
\listoftables
\lstlistoflistings

\chapter{Acknowledgements}

I would like to thank my supervisor, Professor Someone. This
research was funded by the Imaginary Research Council.

\chapter{Abstract}

A brief summary of the project goes here.

\mainmatter

\chapter{Introduction}
\label{ch:intro}

This is an example of how to use Bib\TeX\ and natbib.
A textual citation \citet{turabian96} and a parenthetical citation
\citep[see][Chapter 9]{goossens97}. 

\chapter{Technical Introduction}
\label{ch:techintro}

\section{Listings}

Some sample code is shown in Listing~\ref{lst:sample}.

\lstset{language=C,basicstyle={\ttfamily\singlespacing}}
\begin{lstlisting}[mathescape=true,caption={Sample},label={lst:sample}]
#include <stdio.h> /* needed for printf */
#include <math.h> /* needed for sqrt */

int main()
{
   double x = sqrt(2.0); /* $x = \sqrt{2}$ */

   printf("x = %f\n", x);

   return 1;
}
\end{lstlisting} 

\section{Theorems}

\begin{Definition}[Tautology]
A \emph{tautology} is a proposition that is always true for any
value of its variables.
\end{Definition}

\begin{Definition}[Contradiction]
A \emph{contradiction} is a proposition that is always false for any
value of its variables.
\end{Definition}

\begin{Theorem}
If proposition $P$ is a tautology
then $\sim P$ is a contradiction,
and conversely.
\end{Theorem}

\begin{Example}\label{ex:rain}
``It is raining or it is not raining'' is a tautology,
but ``it is not raining and it is raining'' is a contradiction.
\end{Example}

\begin{Remark}
Example~\ref{ex:rain} used De Morgan’s Law
$\sim (p \vee q) \equiv \sim p \wedge \sim q$.
\end{Remark}

\section{Algorithms}

Using algorithm (theorem-like) and tabbing environments:

\begin{algorithm}[Gauss-Seidel Algorithm]
\begin{tabbing}
1. \=For $k=1$ to maximum number of iterations\\
\>2. For \=$i=1$ to $n$\\
\>\>Set
\begin{math}
x_i^{(k)} =
\frac{b_i-\sum_{j=1}^{i-1}a_{ij}x_j^{(k)}
-\sum_{j=i+1}^{n}a_{ij}x_j^{(k-1)}}%
{a_{ii}}
\end{math}
\\
\>3. If $\lvert\vec{x}^{(k)}-\vec{x}^{(k-1)}\rvert < \epsilon$,
where $\epsilon$ is a specified stopping criteria, stop.
\end{tabbing}
\end{algorithm}

Using floating algorithm2e environment:

\begin{algorithm2e}
\caption{Gauss-Seidel Algorithm}\label{alg:gauss-seidel}
\For{$k\leftarrow 1$ \KwTo maximum iterations}
{
   \For{$i\leftarrow 1$ \KwTo $n$}
   {
     $
      x_i^{(k)} =
      \frac{b_i-\sum_{j=1}^{i-1}a_{ij}x_j^{(k)}
      -\sum_{j=i+1}^{n}a_{ij}x_j^{(k-1)}}%
      {a_{ii}}
      $\;
   }
   \If{$\lvert\vec{x}^{(k)}-\vec{x}^{(k-1)}\rvert < \epsilon$}{Stop}
}
\end{algorithm2e}


\chapter{Method}
\label{ch:method}

The distance was measured in \si{\kilo\metre} and the area in
\si{\kilo\metre\squared}. The acceleration was given in
\si{\metre\per\square\second}.

\chapter{Results}
\label{ch:results}

Out of \num{12890} experiments, \num{1289} of them had a mean
squared error of \num{.346} and \num{128} of them had a mean
squared error of \num{1.23e-6}.

The acceleration was approximately
\SI{9.78}{\metre\per\square\second}.


\chapter{Conclusions}
\label{ch:conc}


\backmatter

% A glossary and list of acronyms may go here
% or may go in the front matter after the abstract.

\bibliographystyle{plainnat}
\bibliography{thesis-ref}

\end{document}
