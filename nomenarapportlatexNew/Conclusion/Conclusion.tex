\chapter*{Conclusion générale}

\qquad Dans ce mémoire, nous avons comparé différentes méthodes d'extraction automatique de termes-clés en évaluant les résultats sur une collection d'articles d'IPM chacun composé du titre et de son résumé.%voulu mesurer la force de liaison entre le titre et le résumé des articles du journal IPM et aussi trouver une meilleur méthode d'extraction automatique de termes-clés sur un document composé que 

%\smallskip
%
%\qquad Nous avons ainsi adopté une mesure très naïve pour savoir la force de liaison d'un titre et de son résumé, mesure naïve dans le sens où nous ne comptons que le nombre de mots du titre qui apparaissent dans le résumé et le nombre d'occurrences de ces mots. Nous avons effectué plusieurs mesures  en tenant compte puis non de l'utilisation des synonymes tel que dans chaque cas, nous avons observé les résultats avant et après les prétraitements ( suppression des mots vides et racinisation). Nous avons pu constater, avec ou sans utilisation des synonymes, que  les mots vides pèsent sur  la mesure de cette force de liaison parce que nous avons observé une diminution de cette mesure. Contrairement à cela, avec l'utilisation de la racinisation, nous avons constaté une augmentation de la mesure la force de liaison.

\smallskip

\qquad Concernant l'extraction des termes-clés, nous avons expérimenté cinq méthodes d'extraction automatiques de termes-clés dont : 
\begin{itemize}
	\item TF-IDF\cite{bougouin2013etat}
	\item TopicRank\cite{bougouin2013topicrank}
	\item TextRank\cite{mihalcea2004textrank}
	\item SingleRank\cite{wan2008single}
	\item K-core\cite{rousseau2015main}
\end{itemize}

\noindent Nous avons constaté que globalement, TF-IDF fournit les termes-clés les plus proches de ceux formulés par les auteurs des articles du fait que cette méthode  utilise tous les documents composant le corpus. Mais TopicRank, qui n'utilise que le document analysé, est l'une des meilleures méthodes d'extraction automatiques de termes-clés: ses performances ne sont pas éloignés de celles de TF-IDF. TopicRank serait d'autant plus performante qu'elle est utilisée sur un document complet au lieu d'un document composé seulement du titre et de son résumé (comme dans notre cas). L'utilisation de la radicalisation sur les documents avant l'extraction des termes-clés ne fournit pas de meilleurs résultats donc son intégration ne s'avère pas être utile dans notre analyse.

\smallskip

\noindent La comparaison des performances en précision de ces méthodes d'extraction automatique de termes-clés sont des comparaisons simples. Des tests statistiques seraient à utiliser dans le but de savoir si la différence de performance entre deux méthodes est significative ou non, surtout entre les méthodes TF-IDF et TopicRank.
%\noindent Nous avons constaté que globalement, TF-IDF fournit les meilleurs résultats du fait que cette méthode  utilise tous les documents composant le corpus. Mais TopicRank, qui n'utilise que le document analysé, est une meilleure méthode d'extraction automatique de termes-clés si nous ne gardons que 33 \% des termes-clés générés. TopicRank serait d'autant plus performante qu'elle est utilisée sur un document complet au lieu d'un document composé seulement du titre et de son résumé (comme dans notre cas). L'utilisation de la radicalisation sur les documents avant l'extraction des termes-clés ne fournit pas de meilleurs résultats donc son intégration ne s'avère pas être utile dans notre analyse.

\smallskip

\noindent Nous nous sommes intéressés à comparer les résultats de d'extraction automatique avec termes choisis par les auteurs; cette méthode pourrait également être utilisée pour suggérer des termes-clés aux auteurs.