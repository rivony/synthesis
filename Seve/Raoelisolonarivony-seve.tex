\documentclass[a4paper,11pt]{moderncv}

\usepackage[french]{babel}
\usepackage[T1]{fontenc}
\usepackage[utf8]{inputenc}
\usepackage{lmodern}
\usepackage{microtype}
\usepackage[margin=2cm]{geometry}

\graphicspath{ {images/} }

\moderncvtheme{classic}
%\setlength{\hintscolumnwidth}{45mm}
%\AtBeginDocument{\recomputelengths}
\name{Raoelisolonarivony}{}
\title{Etudiant}
\address{AB 414 Ampitatafika}{102 ANTANANARIVO}
\phone[mobile]{+261 34 20 410 08}
\email{rivony@gmail.com}
\extrainfo{Célibataire}
\photo{raoelisolonarivony.jpg}

\begin{document}

\makecvtitle

\section{Disciplines et activités}

\subsection{Discipline}
\cvlistitem{Mathématiques}
\cvlistitem{Informatique}
\cvlistitem{Statistique}

\subsection{Activité}
\cvlistitem{Recherche}


\section{Etudes en cours}

\cventry{Depuis Avril 2016}{Préparation du Master 2 en Mathématiques, Informatique et
Statistique Appliquées}
{Faculté des Sciences}{Université d'Antananarivo}
{}{}


\section{Cursus}

\subsection{Faculté des Sciences -- Université d'Antananarivo}
\cventry{2007 -- 2009}{Maîtrise en Informatique et Statistique Appliquées (MISA)}
{}{}{}{}
\cventry{2003 -- 2006}{DUES 2 de Mathématiques}
{}{}{}{}

\subsection{Lycée Moderne Ampefiloha}
\cventry{2002}{Baccalauréat Série C}
{}{}{}{}


\section{Domaine de compétences et expertise}

\subsection{Modélisation et réalisation d’applications d’entreprise}
\cvitem{}{Mise en place de l’architecture des
applications .NET et implémentation des règles et conventions de développement de projets
informatiques durant tout le cycle de vie des applications: depuis l’analyse du besoin
jusqu’à la livraison et l’installation des applications en environnement de production.}

\subsection{Diagnostique et optimisation d’applications}
\cvitem{}{Débogage, observation et expérimentation de
solution d’optimisation des applications en terme d’utilisation de ressources et de temps de
réponse.}

\subsection{Formation}
\cvitem{}{Réaliser et animer des formations sur des sujets se rapportant aux technologies
.Net, incluant la préparation des contenus et des matériels, la formation proprement dite, les
travaux pratiques et l’évaluation des participants.}

\section{Expérience professionnelle}

\cventry{2010 -- 2015}{Analyste-développeur .NET chez Accenture Mauritius Ltd}
{}{}{}{
\begin{itemize}
	\item Analyse des besoins et chiffrage des solutions proposées
	\item Architecture et modélisation des applications
	\item Implémentation, tests et livraison : Assure la qualité des livrables et l’intégrité
fonctionnelle des applications.
	\item Maintenance et évolution : responsable des interventions sur les applications post-
production
	\item Intégration continue : participation et développement d’outils pour la méthode Agile
	\item Lead applicatif : responsable de groupes d’applications, point de contact client
	\item Formateur, avec des participants de niveau de 2\ieme année, 20 heures.
	\item Recherche et développement au sein de la compagnie : Exploration de nouveaux axes de développement de solutions susceptibles d’apporter de la valeur ajoutée aux produits
procurés par la compagnie.
\end{itemize}
}

\cventry{Avril. -- Juin. 2010}{Développeur .NET/iPhone chez eTech-Consulting}
{}{}{}{
\begin{itemize}
	\item Développement de fonctionnalités
	\item Recherche et optimisation d’applications iOS : architecture récursive et couche DAO
\end{itemize}
}

\cventry{Déc. 2009 -- Janv. 2010}{Développeur ASP.NET et PHP chez Ibonia}
{}{}{}{
\begin{itemize}
	\item Développement et maintenance d’applications web
\end{itemize}
}

\cventry{Juin – Sept. 2009}{Consultant informatique chez Profarm}
{}{}{}{
\begin{itemize}
	\item Formateur à l’initiation à Linux, avec des participants médecins et leurs assistants, 120 heures
	\item Mise en place de matériel et logiciel de scan en dentisterie
\end{itemize}
}

\cventry{Oct. 2008 -- Fév. 2009 }{Stage de fin d’étude}
{}{}{}{
\begin{itemize}
	\item Analyste-programmeur au sein d’Air Madagascar. Réalisation de l'application de gestion du programme de fidélisation en ASP.NET « NAMAKO »
\end{itemize}
}

\cventry{Nov. -- Déc. 2007}{Stage de découverte d’entreprise}
{}{}{}{
\begin{itemize}
	\item Analyste-programmeur chez TELMA. Réalisation de l'application de gestion de courrier
de la société en PHP/MySQL dénommé « GESCO »
\end{itemize}
}

\section{Connaissances techniques}

%\subsection{Traitement d'images}
%\cvitem{}{Photoshop??, Gimp??}

%\subsection{Spécifications graphiques numériques}
%\cvitem{}{OPENGL 4??, Vulkan???}

%\subsection{Modélisation 2D/3D}
%\cvitem{}{Maya?, Blender?, 3DS Max??, Realflow???, vray???}

%\subsection{Moteur de jeux vidéos}
%\cvitem{}{Unity?, Unreal Engine?}

\subsection{Développement ASP.NET}
\cvitem{}{ASP.NET MVC (C\#), K2 Workflow, Silverlight}

\subsection{Design pattern}
\cvitem{}{Repository, Dependency injection, Modèle décoration}

\subsection{Langages}
\cvitem{}{C\#, Objective-C, C/C++,
%C/C++11??, 
OCaml, Python 3}

\subsection{Développement web}
\cvitem{}{ASP.NET, HTML5/CSS3/Javascript, Ajax, KnockoutJS, AngularJS, Bootstrap,
PHP/MySQL}

\subsection{Base de données et modélisation}
\cvitem{}{Microsoft SQLServer, Oracle, SQLite, MySQL, UML, Merise}

\subsection{Méthodologie}
\cvitem{}{Microsoft Project, TeamCity}

\section{Langues}

\cvitemwithcomment{Malagasy}{courant}{}
\cvitemwithcomment{Français}{courant}{}
\cvitemwithcomment{Anglais}{intermédiaire}{}

\section{Informations complémentaires}

\subsection{Association AEMISA}
\cvitem{}{Responsable logistique de la Fête de l'Internet à Madagascar éditions 2007 et 2008}

\subsection{AC English Club}
\cvitem{}{Membre de l’American English Club de Tanjombato en 2016}

\subsection{Intérêts/Loisirs}
\cvitem{}{Film d’animations, mécanique automobile, dessin.}

\subsection{Sport}
\cvitem{}{Football, course, basketball, natation, tennis.}

\subsection{Divers}
\cvitem{}{Détenteur d'un permis de conduire - Catégorie B.}

\end{document}