\documentclass[11pt]{article}
\usepackage{geometry}
\geometry{
 a4paper,
 total={170mm,257mm},
 left=20mm,
 top=20mm,
 }
\usepackage[french]{babel}
\usepackage[T1]{fontenc}
\usepackage[utf8]{inputenc}
\usepackage{lmodern}
\usepackage{graphicx}
\usepackage{amssymb}
\usepackage{microtype}
\usepackage[colorlinks=true,
            linkcolor=red,
            urlcolor=blue,
            citecolor=blue]{hyperref}
            
\title{Urbanisation et developpement : les enjeux et defis majeurs dans les cas des villes d'Antananarivo et Moramanga}
\author{ANDRIANJAFIALIVELO Lova Harimanana}
\date{Universite d’Antananarivo - 2013, Maîtrise en économie}

\setlength{\parindent}{2em}
\setlength{\parskip}{1em}
\linespread{1}

\begin{document}

\maketitle

L'urbanisation est un processus de croissance de la population urbaine et d'extension des villes.

Une ville comprend des terrains, des infrastructures, des services, des emplois et des logements. 
Au départ les terrains sont vierges, sauvages ou agricoles. Le premier stade de l'urbanisation consiste à les équiper de routes et d'autres infrastructures. Le deuxième stade est celui des superstructures, lieux de travail et lieux d'habitation et de services.
Les politiques urbaines doivent donc intégrer cinq composantes : le foncier, l'infrastructure, les services, l'emploi et le logement.

Le pourcentage de la population urbaine devrait atteindre $61,8\%$ d'ici au milieu du siècle (2050)\footnote{UN-HABITAT}. La ville d'Antananarivo rassemble plus de 3 millions d'habitants de la population de la grande île.

Les politiques d'urbanisation suivent des plans comme le POS (le plan d'occupation des sols), le PLU (le plan local d'urbanisme) et le PADD (le projet d'aménagement et de développement durables).

L'urbanisme s'appuie généralement sur le réseau de transport et les centres de pôles.

Les avantages sociaux de l'urbanisation sont l'éducation et les opportunités d'emploi.
L'école est la seule institution qui fasse le lien entre les problèmes mondiaux et la vie locale. "L'éducation en vue d'un développement urbain durable" est une stratégie de l'UNESCO pour un environnement urbain viable.

Les effets néfastes de l'urbanisation sur l'environnement vient tout d'abord du fait que la ville est une consommatrice d'espace vitale pour sa construction, son entretien et ses fournitures en énergie. Ensuite, un mouvement de flux continus de déchets solides et liquides y est effectué. La pollution automobile crée des bulles de chaleur. La pollution lumineuse dissuade la faune. La fragmentation des écosystèmes nuit à la reproduction des espèces. Ensuite le risque d'inondations est élevé car le bitume et le béton  qui empêchent la pénétration des eaux de pluies. 
Un autre effet notable est la création de bidonvilles, source d'insalubrité et d'insécurité. La dégradation de la santé publique et le risque d'épidemie élevé.
La congestion urbaine par rapport à la croissance de la population.
L'inégalité sociale avec l'apparition de quartiers résidentiels.

Les entreprises s'installent dans des villes de plus 20 000 habitants pour leur besoin de main d'oeuvre. Pour gérer la sécurité, il devrait y avoir 30 policiers pour ces 20 000 habitants.

Pour les marchés, la proposition de marché à étage est intéressante, l'entreposage des produits et sécurisation des entrepôts.

La ville d'Antananarivo culmine à 1 300 m d'altitude, Sa superficie est de 86,4 $km^2$, $18^{\circ}55^{\prime}$ de latitude sud et $47^{\circ}32^{\prime}$ de longitude. Les paramètre pour générer une ville sont les infos sur les routes et sur la population telles que le nombre, le taux de croissance démographique (Plan d'urbanisme Directeur), le pourcentage d'enfants de moins de 10 ans, entre 11 et 18 ans et les personnes en âge de travailler.


\end{document}