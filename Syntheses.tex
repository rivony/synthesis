\documentclass[11pt]{article}
\usepackage{geometry}
\geometry{
 a4paper,
 total={170mm,257mm},
 left=20mm,
 top=20mm,
 }
\usepackage[french]{babel}
\usepackage[T1]{fontenc}
\usepackage[utf8]{inputenc}
\usepackage{lmodern}
\usepackage{graphicx}
\usepackage{amssymb}
\usepackage{microtype}
\usepackage{hyperref}
\title{Résumé et synthèse d'articles}
\author{Raoelisolonarivony - MISA M2}
\date{Décembre 2016}
\setlength{\parindent}{2em}
\setlength{\parskip}{1em}
\linespread{1.3}
\usepackage{fancyhdr}
\pagestyle{fancy}

\renewcommand{\headrulewidth}{0.5pt}
\fancyhead[L]{RAOELISOLONARIVONY - MISA M2 - \textit{Synthèse de documents}}
\fancyhead[C]{}
\fancyhead[R]{Décembre 2016}

\begin{document}

\section{Façon 2}

\subsection{Titre}

\emph{\large Les aspects de la publicité sur le plan social, musical et médiatique.}

\subsection{Résumé 1}

La société d'assurance SAAQ a répandu des publicités sanglantes. Est-il utile et éthique d'inspirer l'effroi? 

D'après le professeur Claude Cossette, les publicitaires agissent mécaniquement en ignorant les retombées négatives. Contrairement aux médecins, suivre une éthique n'est pas leur priorité. Ils servent d'intermédiaires entre les annonceurs et le public, seul le profit compte. La formation des publicitaires se focalise sur les techniques, les valeurs morales sont délaissées. 

Les publicités sensibilisant par la peur traitent des fléaux sociaux. Si la peur réveille, elle ne persuade pas, elle pousse au désintéressement. Comme les organismes annonceurs se soucient de la jeunesse, les publicitaires doivent préserver ces âmes sensibles des dommages de la peur. 

Il est immoral d'exploiter des enfants démunis pour une propagande électorale. 

(\textbf{120 mots})

\subsection{Résumé 2}

Ce n'est pas nouveau pour les publicitaires d'acquérir une licence d'utilisation d'\oe uvres musicales. Parfois, la musique et la publicité sont créées de manière synchrone.

Historiquement, les régies de la radio ont adopté la musique dans les annonces publicitaires comme la musicalisation de communiqués assommants chez Radio-Ciné par Charles Trénet.
En 1968, la France autorise la diffusion de publicités de marque et favorise les projets publicitaires télévisés impliquant des musiciens comme Serge Gainsbourg.
Ces artistes parodiaient leurs créations en les adaptant pour la publicité. Certaines compositions, intégralement dédiées à la publicité, deviennent des succès comme "Ticket to Ride" des Beattles.
 
Le détenteur des droits des titres doit être un partenaire. Les maisons de disques créent des départements pour gérer l'utilisation publicitaire de leurs propriétés et vendre les droits dérivés aux publicitaires.

(\textbf{130 mots})

\newpage
\subsection{Résumé 3}

Les médias font de la publicité pour diverses raisons.

Premièrement, la structure des coûts du produit médiatique se distingue par une économie d'échelle et des coûts à faible variation. Pour produire les quotidiens, le prototypage coûte très cher et c'est un coût fixe. Les coûts fixes des médias sont irrécupérables : la presse et les programmes télévisés nécessitent un investissement conséquent en temps et en argent sans aucune garantie de réussite.

Deuxièmement, la masse média est un bien "public". C'est un bien non-rival: des agents arbitraires peuvent simultanément consommer le même produit médiatique. De plus, les médias sont non-exclusifs: tout le monde peut en profiter sans payer. Les biens semi publics tarifie leur prestation comme les chaînes payantes.
Par conséquent, vu son aspect non-rival, un produit unique destiné à plusieurs consommateurs ne favorise pas la pluralité des goûts en matière d'information. Etant publics et non-exclusifs, les médias sont contraints de trouver les ressources pécuniaires par les publicités en plus des redevances sur les industries gouvernementales liées à la communication.

Troisièmement, les médias sont des biens tutélaires. Leur utilité dépend du consentement de l’État qui peut décider de privilégier des campagnes médiatiques dans son intérêt. En contraste, pour le respect de la démocratie, l'Etat favorise la liberté journalistique et exhorte à la libre expression d'opinions politiques en subventionnant les médias. Particulièrement, l'Etat soutient financièrement la presse  pour alléger les coûts de production des journaux et assurer l'accès à l'information au grand public.
L'Etat et les annonceurs sont des acteurs principaux sur le secteur de la publicité médiatique.

Les industriels déboursent des sommes colossales dans les médias pour promouvoir leurs produits. Tous les différents types de médias sont sollicités: les médias ne touchant pas de recettes publicitaires comme Charlie Hebdo ou Arte, les médias dépendant totalement de la publicité comme M6, et les médias tirant ses bénéfices à la fois de la publicité et des recettes de ventes. Ce dernier type regroupe la majorité des médias français. Les opérateurs publics sont des cas particuliers bénéficiant de la publicité et des redevances payées par les téléspectateurs.

Une analyse de Bipe datant de 2004 indique que la France est en retard dans le marché de la publicité comparée à ses pays voisins. Durant les vingt dernières années, deux périodes caractérisent l'évolution de la publicité en France: la première sous l'impulsion de la télévision début 1980 et la deuxième après une crise en 1990. Elles ont apporté une forte croissance des investissements industriels sur les médias jusqu'à la toile internet en 2001.

Au final, les médias tirent leur financement de la publicité. S'ils dépendent des annonceurs pour leur profit, les annonceurs quant à eux gagnent la possibilité de propager leurs annonces. Les médias produisent un travail destiné à la fois aux annonceurs (environnement de diffusions) et aux usagers (des programmes intéressants, le journal, la météo).

(\textbf{470 mots})

\newpage
\subsection{Introduction}

Selon un proverbe chinois, "les bonnes marchandises se passent de publicité". Pourtant, il suffit d'ouvrir les yeux, de tendre l'oreille pour constater que la société est submergée par la publicité. Quoiqu'il en soit, selon un article de \underline{Pauline Gravel}, le publicitaire doit avoir une ligne de conduite professionnelle et penser aux conséquences de son message sur ses prospects. Une bonne publicité n'effraye pas, même si elle cherche à persuader. La publicité doit attirer l'attention du consommateur et susciter l'envie. D'après \underline{Christophe Magis}, les musiciens et leur création sont mis à contribution pour retenir l'intérêt des usagers. Pour diffuser les publicités, le support par excellence est la masse média. Pour \underline{Nathalie Sonnac}, les médias et la publicité sont liés par une interdépendance indéfectible, une survie commune.

Cette note rassemble trois articles en ligne discutant du besoin d'éthique professionnelle chez les publicitaires, du rôle que la musique joue dans l'amélioration des annonces publicitaires et de l'interaction étroite qui existe entre les médias et la publicité.

\subsection{Conclusion}

La publicité est un bon moyen pour atteindre les consommateurs. Pour les faire réagir, certains publicitaires déclenchent des émotions comme la peur ou le dégoût. Ces techniques choquantes témoignent de leur avidité et de leur manque de moralité. Dans "\textit{L'utilisation de la peur en publicité - L'éthique est absente du vocabulaire des publicitaires}", publié en 2008, \underline{Pauline Gravel} évoque le sujet de l'éthique devant l'inconscience de certains publicitaires exploitant la peur comme un moteur de persuasion, entrainant des dégâts collatéraux. La profession de publicitaire doit être responsabilisée en se fixant un code de conduite, surtout pour épargner les âmes sensibles.

Pour charmer les consommateurs, la publicité s'est trouvée une grande alliée en la musique. D'après \underline{Christophe Magis}, dans son article intitulé "\textit{Musique et publicité : les enjeux de la synchronisation}, publié en janvier 2014, la musique a apporté du rythme aux annonces radiophoniques ennuyeux à l'origine. Plus tard, elle fait vibrer les téléspectateurs dans les spots publicitaires. Les créations musicales associées à la publicité sont transformées ou prises dans leur intégralité. Les artistes cèdent leur droit moyennant une compensation. A plus grande échelle, les maisons de production de musique chargent un département spécifique pour gérer les droits d'auteur et de reproduction de leurs titres.

La publicité et les médias sont pratiquement inséparables. Selon \underline{Nathalie Sonnac}, dans son livre "\textit{Médias et publicité ou les conséquences d’une interaction entre deux marchés}, publié en 2006, les médias se reposent sur les recettes publicitaires pour combler leur coût de production exorbitants, leur caractère "public" et non-exclusif. Quant à la publicité, elle a besoin des supports de diffusion audiovisuels pour atteindre le maximum de prospects. Les industries investissent énormément dans les médias pour faire connaître leurs produits et leurs services. Les médias procurent à la fois des informations pour les consommateurs et un outil de diffusion pour les publicitaires.

\newpage

\section{Façon 1}

\subsection{Titre}

\emph{\large Les aspects de la publicité sur le plan social, musical et médiatique..}

\subsection{Liste des documents}
Document I: \textit{L'utilisation de la peur en publicité - L'éthique est absente du vocabulaire des publicitaires}, Pauline Gravel, www.ledevoir.com, publié le 9 mai 2008

Document II: \textit{Musique et publicité : les enjeux de la synchronisation}, Christophe Magis,\newline www.inaglobal.fr, publié le 7 janvier 2014

Document III: \textit{Médias et publicité ou les conséquences d’une interaction entre deux marchés}, Nathalie Sonnac, Le Temps des médias, 1/2006 ($n^{\circ}$ 6), p. 49-58 


\subsection{Synthèse}

Cette note rassemble les sujets abordés dans ces trois documents et fait état de l’implication de la publicité dans la société, la musique et les médias audiovisuels.

 Sur le plan social, \underline{Pauline Gravel} relate les graves répercussions de la publicité inspirant la peur sur la jeunesse et les personnes âgées. Elle dénonce leur manque d’éthique professionnelle dans leur production. Ces publicitaires, selon elle, ne songent qu’au profit.

D’après \underline{Nathalie Sonnac}, la publicité est diffusée via des supports médiatiques aux coûts de production élevés. Ces médias sont confrontés à des coûts fixes et soumis à une économie d'échelle. Le retour sur investissement est hasardeux pour ces activités. \underline{Christophe Magis} ajoute que les publicitaires forment des partenariats avec des musiciens pour améliorer leurs annonces. L’acquisition de licences d’utilisation d’une \oe uvre musicale coûte chère. La preuve est que les maisons de disques se concentrent sur la commercialisation des droits d’auteurs très profitable pour l’industrie musicale.

D’un autre côté, \underline{Nathalie Sonnac} signale que les industries et l’Etat sont les principaux annonceurs de publicités. Ils investissent généreusement dans les médias. Les industriels veulent à tout prix faire connaitre leurs produits aux consommateurs. Quant à l’État, les médias sont sous sa tutelle, il peut à la fois les censurer dans leurs publications et financer les programmes en faveur de la démocratie et de la liberté d’expression. Les publicitaires y trouvent donc leur avantage puisque les demandes de leur service augmentent en conséquence.

Seulement, toujours d’après \underline{Nathalie Sonnac}, les médias sont à caractère public et non-exclusif, les publicités diffusées sur les médias touchent tout le public sans exception. \underline{Pauline Gravel} souligne le cas des campagnes publicitaires initiées par les organismes sociaux et destinées aux jeunes. Même si les jeunes sont avides de sensations fortes, les messages transmis doivent être murement réfléchis en prenant en compte l’audience. \underline{Pauline Gravel} insiste sur la nécessité de la responsabilisation des publicitaires, ces derniers se comportant juste comme des intermédiaires entre les annonceurs et le public.

La publicité doit faire envie aux prospects. D’après \underline{Christophe Magis}, les régies des radios ont été les premières à introduire la musique dans leurs annonces. Selon lui, les musiciens ont été mis à contribution pour remédier à la monotonie des publicités. Des \oe uvres musicales devenues célèbres par la suite ont été composées intégralement à des fins publicitaires. Il fait remarquer que la France a démarré en retard dans cette synchronisation entre la musique et la publicité. La France est aussi à la traîne, selon \underline{Nathalie Sonnac}, en ce qui concerne l’adoption de la publicité dans les médias télévisés.

\underline{Nathalie Sonnac} conclue que les médias dépendent financièrement en grande partie des annonceurs. Réciproquement, les annonceurs ont besoin des supports publicitaires pour faire entendre leur voix. Les médias offrent au le public des programmes d'information et de divertissement non-exclusifs.
 Tout bien considéré, selon \underline{Pauline Gravel}, la publicité est destinée à être un outil efficace pour persuader les consommateurs de manière plaisante à condition d'éviter les manquements éthiques. \underline{Christophe Magis} montre que les publicitaires disposent d’atouts non négligeables pour améliorer leur travail comme la musique et \underline{Nathalie Sonnac} parle des appuis financiers de l’Etat et des entreprises envers les médias pour réduire les coûts de production.

(\textbf{500 mots})

\newpage

\end{document}

