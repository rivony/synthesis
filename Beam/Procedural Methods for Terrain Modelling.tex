\documentclass{beamer}

\usepackage[french]{babel}
\usepackage[T1]{fontenc}
\usepackage[utf8]{inputenc}
\usepackage{lmodern}
\usepackage{microtype}

\setbeamercovered{transparent=20}
\usetheme{AnnArbor}
%\usecolortheme[named=red]{structure}

\title{Recensement des méthodes procédurales \\
pour la modélisation de terrain}
\subtitle{Etat de l'art}
\author{Ruben M. Smelik, Klaas Jan de Kraker, Saskia A. \\
Groenewegen, Tim Tutenel, Rafael Bidarra}
\institute{TNO Defence, Security and Safety, The Hague, The Netherlands \\
Delft University of Technology, Delft, The Netherlands}
\date{2009}

\begin{document}

\begin{frame}
\titlepage
\end{frame}

%-------------------------Frame 2: Plan---------------------------------------
\begin{frame}{Plan de l'article}
\begin{itemize}
\item <1-> Abstract
\item <2-> Introduction
\item <3-> Height-maps
\item <4-> Rivières, Océans et Lacs
\item <5-> Les modèles de plantes et la distribution de la végétation
\item <6-> Les réseaux routiers
\item <7-> Les environnements urbains
\item <8-> Conclusions
\end{itemize}
\end{frame}

%-------------------------Frame 3: Abstract---------------------------------------
\begin{frame}
\frametitle{Abstract}

Les \textbf{méthodes procédurales} sont des alternatives pour la modélisation manuelle d'environnements.\newline

\begin{uncoverenv}<2->
Les points négatifs de ces méthodes sont principalement:
	\begin{itemize}
	\item l'imprévisibilité et le manque de contrôle sur les résultats
	\item l'absence de solutions intégrées\newline
	\end{itemize}
\end{uncoverenv}

\uncover<3->{Cette note recense les méthodes procédurales appliquées sur la \textbf{modélisation de terrain}.}\newline

\begin{uncoverenv}<4->
Elle évalue:
\begin{itemize}
	\item le réalisme de leur résultat
	\item la performance
	\item l'interactivité de la procédure avec les utilisateurs
\end{itemize}
\end{uncoverenv}
\end{frame}

%-------------------------Frame 4: Introduction - 1/6----------------------------
\begin{frame}
\frametitle{Introduction}
Les mondes virtuels 3D sont complexes. 

\begin{uncoverenv}<2->
Le processus, les outils et les techniques de modélisation de ces mondes sont:
\begin{itemize}
	\item \textbf{laborieuses} et \textbf{répétitives}
	\item nécessitent des qualités spéciales pour la modélisation 3D \newline
\end{itemize}
\end{uncoverenv}

\uncover<3->{
La modélisation procédurale consiste à \textbf{créer ses contenus automatiquement} au lieu de les modéliser à la main.}\newline

\begin{uncoverenv}<4->
Cette approche a eu du succès pour générer, par exemple:
\begin{itemize}
	\item les textures
	\item les modèles géométriques
	\item les animations
	\item les sons...\newline
\end{itemize}
\end{uncoverenv}

\end{frame}

%-------------------------Frame 5: Introduction -2/6----------------------------
\begin{frame}{Introduction (2/6)}

Un sujet majeur dans la modélisation procédurale est la \textbf{génération automatique de modèles de terrain}, pour:
\begin{itemize}
	\uncover<2->{
	\item les phénomènes naturels (datant des années 80 et 90):
		\begin{itemize}
			\item les élévations de terrain
			\item la croissance des plantes
		\end{itemize}
	}
	\uncover<3->{
	\item les environements urbains (début du nouveau millénaire)
	}\newline
\end{itemize}

\begin{uncoverenv}<4->
La transition de la modélisation manuelle vers la modélisation automatisée est limitée par le fait que:
\begin{itemize}
	\item les articles de recherche autant que les outils commerciaux \textbf{se concentrent sur un seul aspect de la modélisation de terrain} (par exemple, la génération de profils d'élévation d'intérêt) et gère les aspects restants à un degré limité ou pas du tout
\end{itemize}
\end{uncoverenv}

\end{frame}

%-------------------------Frame 6: Introduction -3/6----------------------------
\begin{frame}{Introduction (3/6)}

\begin{itemize}
	\item l'intégration et l'ajustement des méthodes procédurales pour qu'elles puissent automatiquement générer un model de terrain complet et consistent demeurent \textbf{jusqu'à maintenant irrésolus}
	\item le manque de contrôle qu'elles procurent. L'aléa inhérent au résultat obtenu force souvent les utilisateurs à modéliser par "trial and error" (generate and test).
	\item Pas assez reçu encore d'attention accordée à la résolution des problèmes de ces méthodes. \newline
\end{itemize}

\uncover<2->{Les propriétés importantes de ces méthodes:}

\begin{uncoverenv}<3->
\begin{itemize}
	\item le niveau de réalisme du résultat
	\item la performance de l'algorithme
	\item les facilités qu'il offre aux utilisateurs sur le contrôle du processus de génération\newline
\end{itemize}
\end{uncoverenv}

\end{frame}

%-------------------------Frame 7: Introduction -4/6----------------------------
\begin{frame}{Introduction (4/6)}
L'objectif de cette note est double:

\begin{itemize}
	\item<2-> donner aux lecteurs intéressés par le domaine de la modélisation de terrain procédurale une vue d'ensemble des recherches faites à ce jour
	\item<3-> donner les possibilités de résolution des problèmes connus liés aux modèles de terrain en cours de traitement et ce qui reste à faire.\newline
\end{itemize}

\uncover<4->{
\textbf{Smellik et al., 2008, 2009}: conditions à remplir qu'un framework de modélisation procédurale doit remplir pour être qualitativement acceptable et représenter une alternative productive au workflow de modélisation existant.
}\newline

\uncover<5->{Nous décrivons le design conceptuel d'un tel framework.}\newline

\end{frame}
	
%-------------------------Frame 8: Introduction -5/6----------------------------	
\begin{frame}{Introduction (5/6)}

Il intègre les modélisations procédurales et gère les dépendances entre les particularités des terrains dans le but de générer un modèle de terrain complet et consistent, qui correspond à un \textit{schéma grossier dessiné du terrain fait par l'utilisateur}.\newline

\begin{uncoverenv}<2->
Il concerne la distinction de plusieurs \textbf{couches} dans le modèle du terrain, chacune contenant des caractéristiques:
\begin{itemize}
	\item naturelles (terre, eau, couches de végétation)
	\item faites par l'homme (route et couches urbaines)
\end{itemize}
\end{uncoverenv}

\uncover<3->{Nous présentons le design, l'implémentation et les résultats de deux de ces couches.}\newline

\uncover<4->{Dans cette note, la distinction des couches du terrain est aussi pratique pour  structurer le travail que nous recensons.}\newline

\end{frame}

%-------------------------Frame 8: Introduction -6/6----------------------------
\begin{frame}{Introduction (6/6)}

Les sections suivantes discutent des méthodes procédurales pour:
\begin{itemize}
	\item[1] les données d'élévation
	\item[2] les corps d'eau
	\item[3] la végétation
	\item[4] les réseaux routiers
	\item[5] les environnements urbains
\end{itemize} 
suivi d'une concludion de l'état de l'art.

\end{frame}

%-------------------------Frame 9: Height-maps 1/12----------------------------
\begin{frame}{Height-maps}

Les Height-maps sont des grilles à deux dimensions contenant des valeurs d'élévation.\newline

\uncover<2->{Elles sont souvent utilisés comme la base d'un modèle de terrain}\newline

\uncover<3->{Il y a beaucoup d'algorihtmes procédurales pour créer les height-maps.} \newline

\uncover<4->{Parmi les plus anciens algorithmes figurent les méthodes \textbf{basées sur la subdivision}.}

\end{frame}

%-------------------------Frame 10: Height-maps 2/12 ----------------------------
\begin{frame}{Height-maps (2/12)}

Les méthodes \textbf{basées sur la subdivision}. \newline

\uncover<2->{Un height-map grossier est itérativement subdivisé, chaque itération utilise un aléa controllé pour ajouter les détails.} \newline

\uncover<3->{Miller (1986) décrit plusieurs variantes de cette méthode de \textbf{déplacement de point de milieu (midpoint)} dans laquelle un nouveau point d'élévation est assigné à la moyenne de ces coins dans un triangle ou une forme de diamant additionnée d'une correction aléatoire.} \newline

\uncover<4->{L'intervalle de la correction diminue à chaque itération dépendant d'un paramètre qui contrôle la rugosité du height-map résultant.} \newline

\end{frame}

%-------------------------Frame 11: Height-maps 3/12 ----------------------------
\begin{frame}{Height-maps (3/12)}

La génération height-maps est de nos jours souvent basée sur les \textbf{générateurs de bruits fractals} (Fournier et al., 1982; Voss, 1985), comme le \textbf{bruit de Perlin} (Perlin 1985), qui génère des bruits en échantillonant et en interpolant des points de la grille de vecteurs aléatoires.\newline

\uncover<2->{Redimensionner et ajouter plusieurs niveaux de bruit pour chaque point du height-map conduit à des structures naturelles, pseudo montagneux.}\newline

\uncover<3->{Un livre recommandé pour les bruits fractals et la génération de height-map et celui d'\textbf{Ebert et al. (2003)}.}\newline

\uncover<3->{Ces height-maps \textit{peuvent être transformés} sur la base de filtres communs d'image (ex. lissage) ou sur les simulations de phénomènes physiques, comme l'\textbf{érosion}.}

\end{frame}

%-------------------------Frame 12: Height-maps 4/12 ----------------------------
\begin{frame}{Height-maps (4/12)}

L'\textbf{érosion thermique} réduit les \textbf{changements aigus} pour l'élévation, en distribuant itérativement le matériau depuis les points les plus hauts vers les plus bas, jusqu'à ce qu'un angle de talus, i.e. un \textit{angle maximale de stabilité du matériau} comme pour la pierre ou le sable, soit atteint. \newline

\uncover<2->{L'\textbf{érosion causé par les pluies} (érosion fluviale) peut être simulée en utilisant, par exemple, l'\textbf{automate cellulaire}, où la quantité d'eau et de matériel dissout qui s'écoule en dehors des  autres cellules est calculée sur la base de la descente (pente) locale de la surface du terrain.} \newline

\uncover<3->{Musgrave traite \textit{tous les types d'érosion} (Musgrave et al., 1989; Musgrave, 1993) et Olsen (2004) discutent de plusieurs \textit{optimisations de calcul} avec une qualité réduite mais acceptable.}

\end{frame}

%-------------------------Frame 13: Height-maps 5/12 ----------------------------
\begin{frame}{Height-maps (5/12)}

Benes et Forsbach (2001) introduisent une \textit{structure de terrain convenable pour des algorithmes d'érosion} plus réaliste. \newline

\uncover<2->{Leur modèle de terrain consiste en des tranches de matériaux empilées horizontalement, chacune ayant des valeurs d'élévation et des propriétés des matériaux, ex. la densité.} \newline

\uncover<3->{C'est un compromis entre la structure height-map limitée mais efficace et le terrain tout en voxel.} \newline

\uncover<4->{Le modèle permet également d'avoir des couches d'air, par ce moyen il supporte des structure caverneux (des grottes).}

\end{frame}

%-------------------------Frame 14: Height-maps 6/12 ----------------------------
\begin{frame}{Height-maps (6/12)}

Bien que ces algorithmes d'érosions \textit{ajoutent beaucoup de réalisme} aux terrains montagneux, ils sont réputés \textbf{lents}, ayant besoin de faire des \textit{centaines, voire des milliers d'itérations.} \newline

\uncover<2->{De récentes recherches se sont penchées sur des \textbf{algorithmes d'érosion interactifs}, souvent en portant les algorithmes sur le GPU. Des exemples prométeuses comprennent (Anh et al. 2007) et (Stava et al., 2008).} \newline

\uncover<3->{La génération de height-map basée sur le bruit basique produit des résultats suffisament aléatoires; les utilisateurs contrôlent le résultat seulement à un niveau global, souvent utilisant des paramètres non intuitifs.}

\end{frame}

%-------------------------Frame 15: Height-maps 7/12----------------------------
\begin{frame}{Height-maps (7/12)}

Plusieurs chercheurs se sont penchés sur ce problème. \newline

\begin{uncoverenv}<2->
Stachniak and Stuerwlinger (2005) propose une méthode qui intègre les \textbf{contraintes} (exprimées comme des masques d'images) dans le processus de génération de terrain.
\end{uncoverenv}

\uncover<3->{Il emploie un algorithme de recherche qui rassemble un \textbf{ensemble d'opérations de déformation} acceptable à appliquer à un terrain aléatoire dans le but d'obtenir un terrain qui est conforme à ces contraintes.}\newline

\begin{uncoverenv}<4->
Schneider et al. (2006) introduisent un environnement d'édition dans lequel l'utilisateur peut changer le terrain en modifiant interactivement les fonctions de base du générateur de bruits (en remplaçant le bruit de Perlin par un \textit{ensemble d'images en niveau de gris dessinées par l'utilisateur})
\end{uncoverenv}

\end{frame}

%-------------------------Frame 16: Height-maps 8/12----------------------------
\begin{frame}{Height-maps (8/12)}

Zhou et al. (2007) décrivent une technique qui génère le terrain \textbf{basé sur un exemple en entrée height-map et une ligne dessinée par l'utilisateur} qui définie l'occurence de caractéristiques de ligne courbée en niveau de gris, comme une crête de montagne.

\uncover<2->{Les caractéristiques sont extraites de l'exemple height-map et doivent correspondre aux courbes dessinées et être recousus sur le height-map résultant.} \newline


\uncover<3->{De Carpentier et Bidarra (2009) introduisent les \textbf{pinceaux procéduraux} (procedural brushes)}\uncover<4->{: les utilisateurs peignent le terrain en height-map directement en 3D en appliquant simplement les pinceaux élévante de terrain (terrain raising brushes) mais également des pinceaux basés sur la GPU qui génère plusieurs types de bruit en temps réel (voir Fig. 1a).}

\end{frame}

%-------------------------Frame 17: Height-maps 9/12----------------------------
\begin{frame}{Height-maps (9/12)}

Saunders (2006) propose une méthode très différente, laquelle synthétise un height-map en \textbf{se basant sur des Modèles d'élévation digitale} (ou DEM: Digital Elevation Models) du \textbf{terrain réel}.\newline

\uncover<2->{Un utilisateur de son système \textit{Terrainosaurus} dessine un carte 2D de régions polygonales, chacune de ces régions est marquée pour avoir un certain profil d'élévation. Pour le réalisme, les bords directs (straight boundaries) de la région sont perturbés puis rastérisés dans une grille.} \newline

\uncover<3->{Un height-map est instantié en utilisant un algorithme génétique qui sélectionne les données DEM qui les fait correspondre au profil d'élévation requis.}

\end{frame}

%-------------------------Frame 18: Height-maps 10/12----------------------------
\begin{frame}{Height-maps (10/12)}

Kamal et Uddin (2007) présente un algorithme de déplacement en milieu de point sous contrainte (\textbf{constrained mid-point displacement algorithm}) qui crée une seule montagne correspondante à des propriétés comme l'élévation et l'étalement de la base. \newline

\uncover<2->{Belhadj (2007) introduit un système plus général où un ensemble de valeurs élévations connues contraignent le processus de déplacement en milieu de point (mid-point deplacement).}\newline

\uncover<3->{Des applications possibles sont l'interpolation grossière et incomplète de DEM ou des lignes dessinées par l'utilisateur.} \newline

\end{frame}

%-------------------------Frame 19: Height-maps 11/12----------------------------
\begin{frame}{Height-maps (11/12)}

Une limitation inhérente aux height-map est qu'ils \textit{ne supportent pas les rochers surplombants et les grottes (cavités).}\newline

\uncover<2->{Gamito et Musgrave (2001) proposent un \textbf{système de gauchissement de terrain} qui résulte à un surplomb régulier artificiel.}\newline

\uncover<3->{Une méthode récente (Peytavie et al., 2009) procure une structure plus élaborée avec \textbf{différentes couches de matériaux qui supportent les rochers, les arches, les surplombs et les grottes.}}\newline

\uncover<4->{Leurs modèles de terrain sont visuellement plausibles et naturels.}\newline

\end{frame}
%-------------------------Frame 20: Height-maps 12/12----------------------------
\begin{frame}{Height-maps (12/12)}

Comme illustration de l'état-de-l'art parmi les outils de support, (Voir Fig. 1d) pour un rendu d'un height-map généré par \textbf{L3DT (Torpy, 2009)}, un des outils commerciaux pour la création de height-maps.

\end{frame}

%-------------------------Frame 21: Rivières, Océans et Lacs ---------------------------
\begin{frame}{Rivières, Océans et Lacs}

Pour générer des rivières, plusieurs auteurs ont proposés des algorihtmes qui s'exécutent soit pendant soit après la génération height-map.\newline

\uncover<2->{Kelley et al (1988) prennent un réseau de rivière comme la base d'un height-map. Ils commencent par une unique rivière bien droite et la subdivisent récursivement, produisant un réseau de courants.}\newline

\uncover<3->{Cela forme un squelette pour le height-map, qui est complété en utilisant une fonction d'interpolation de données éparpillées.}\newline

\uncover<4->{Le type de \textbf{climat} et le \textbf{matériau du sol} influencent la forme du réseau du courant.} \newline

\end{frame}

%-------------------------Frame 22: Rivières, Océans et Lacs (2/5)----------------------
\begin{frame}{Rivières, Océans et Lacs (2/5)}

Prusinkiewicz et Hammel (1993) combinent la génération d'une rivière courbée avec un schéma de subdivision de height-map.

\begin{itemize}
	\item<2-> Du triangle de départ d'une rivière, un bord est marqué comme l'entrée et l'autre bord comme la sortie d'une rivière.
	
	\item<3-> Dans l'étape de subdivision, le triangle est divisé en triangles plus petites, et la course de la rivière à partir de l'entrée jusqu'à la sortie peut prendre maintenant plusieurs formes alternatives.
	
	\item<4-> L'élévation des triangles contenant la rivière est prise comme la somme des déplacements négatives de la rivière sur tous les niveaux de récursion (produisant le lit de la rivière), les autres triangles sont traités en utilisant le déplacement de milieu de point standard.
	
	\item<5-> Après huit ou plus récursions, la course de la rivière résultante parait raisonnablement naturelle.
\end{itemize}

\end{frame}

%-------------------------Frame 23: Rivières, Océans et Lacs (3/5)-----------------------
\begin{frame}{Rivières, Océans et Lacs (3/5)}

Un point négatif majeur de l'approche est que la rivière est placée à un \textbf{niveau d'élévation constant et bas}, et alors les cisailements s'enfoncent \textit{à travers un terrain montagneux de manière non naturelle.} \newline

\uncover<2->{Une approche plus avancée décrite par Belhadj et Audibert (2005) crée un height-map avec les crêtes de montagne et les réseaux de rivières.}

\begin{itemize}
	\item <3-> Commençant avec une carte vide, ils placent les paires de particules de crête sur une élévation particulièrement haute et les déplacent dans les directions opposées pendant plusieurs itérations.
	\item <4-> Une courbe de Gauss est dessinée sur le height-map le long des positions de la particule pour chaque itération.
\end{itemize}

\end{frame}
%-------------------------Frame 24: Rivières, Océans et Lacs (4/5)-----------------------
\begin{frame}{Rivières, Océans et Lacs (4/5)}

\begin{itemize}
	\item Ensuite, ils placent les particules de la rivière suivant le sommet de la crête de la montagne et les laissent s'écouler vers le bas suivant des règles simples de la physique (comparable à l'érosion hydraulique).
	\item <2-> Le points restants entre les crêtes et les rivières sont complétés avec une technique de déplacement de point milieu inversé.
\end{itemize}

\uncover<3->{Pour ce type de terrain spécifique, i.e. les crêtes de montagnes et les vallées irriguées par un réseau de rivière dense, la méthode est rapide et efficace.} \newline

\end{frame}

%-------------------------Frame 25: Rivières, Océans et Lacs (5/5)-----------------------
\begin{frame}{Rivières, Océans et Lacs (5/5)}

A l'exception des rivières, les corps d'eau procédurale, comme les océans et les lacs et leurs connections, réseaux de courants, deltas et chutes d'eau, ont retenu peu d'attention jusqu'à maintenant.\newline

\uncover<2->{La formation des lacs n'est pas considérée du tout.}\newline

\uncover<3->{Les océans sont communément générés en  prenant un niveau fixe d'eau (ex. $0m$) ou en commençant par un algorithme d'inondation à partir des points à faible élévation.}\newline

\uncover<4->{Teoh (2008) a également déclaré que la recherche sur ce domaine est incomplète: plusieurs rivières et caractéristiques côtières n'ont pas été traitées.}\newline

\uncover<5->{Il propose des algorithmes simples et rapides pour les rivières sinueux, les deltas et la formation de la plage, qui requiert plus de travail pour augmenter le réalisme.}

\end{frame}

%------Frame 26: Les modèles de plantes et la distribution de la végétation (1/5)--------
\begin{frame}{Modèles de plantes et Distribution de la végétation}

A propos de la végétation, des auteurs ont développé plusieurs :
\begin{itemize}
	\item <2-> procédures de génération d'arbres et de modèles de plantes
	\item <3-> méthodes de placement automatique de la végétation
\end{itemize}
\uncover<4->{dans un modèle de terrain.}\newline

\uncover<5->{Les premières sont utilisées pour obtenir rapidement un ensemble de modèles de plantes similaires mais variables d'une même espèce.}\newline

\uncover<6->{Les secondes délestent les modeleurs de terrain de la tâche laborieuse de placer manuellement tous ces modèles de végétation individuels dans une forêt large.}

\end{frame}

%------Frame 27: Les modèles de plantes et la distribution de la végétation (2/5)--------
\begin{frame}{Modèles de plantes et Distribution de la végétation (2/5)}

Les modèles de plantes procéduraux croissent, en commençant à la racine, ajoutant de plus en plus de petites branches et terminant par les feuilles. \newline

\uncover<2->{Ils sont fondés sur la \textbf{grammaire formelle} (grammar rewriting).} \newline

\uncover<3->{Prusinkiewicz et Lindenmayer (1990) discutent du Lindenmayer-system, ou \textbf{L-system}, un système de réécriture souvent utilisé.}

\uncover<4->{Ils expliquent comment les règles de production peuvent être appliquées en 3D, et présentent plusieurs exemples d'arbres générés ensemble avec leur grammaire.} 

\end{frame}

%------Frame 28: Les modèles de plantes et la distribution de la végétation (3/5)--------
\begin{frame}{Modèles de plantes et Distribution de la végétation (3/5)}

Lintermann et Deussen (1999) proposent un système plus intuitif  pour modéliser procéduralement les plantes, en plaçant \textbf{les composantes de la plante (ex. une feuille) dans un graphe.} \newline

\uncover<2->{Les composantes connexes peuvent être structurées dans un sous-graphe (ex. une brindille).}\newline

\uncover<3->{Le système parcours le graphe, générant et plaçant les instances des composantes dans un graphe intermédiaire qui est utilisé pour la génération géométrique.}\newline

\uncover<4->{La figure 1e) montre un arbre créé avec leur logiciel de modélisation de plante commercial XFrog.}

\end{frame}

%------Frame 29: Les modèles de plantes et la distribution de la végétation (4/5)--------
\begin{frame}{Modèles de plantes et Distribution de la végétation (4/5)}

Deussen et al. (1998) décrivent un  modèle de simulation d'écosystème pour remplir une surface de végétation.\newline

\uncover<2->{L'entrée pour le modèle de simulation est le height-map et une carte de l'eau, plusieurs propriétés d'espèces de plantes, comme le taux de croissance, et, optionnellement une distribution initiale de plantes.}\newline

\uncover<3->{Basé sur cela et en prenant en compte  les règles sur la compétition du sol, des rayons de lumière et de l'eau, \textbf{une distribution de plantes à l'intérieur d'une surface est itérativement déterminée} (voir Fig. 1b), s'exécutant pendant plusieurs minutes. }

\end{frame}

%------Frame 30: Les modèles de plantes et la distribution de la végétation (5/5)--------
\begin{frame}{Modèles de plantes et Distribution de la végétation (5/5)}

Une autre procédure de placement de la végétation par Hammes (2001) est \textbf{fondée sur les écosystèmes.}\newline

\uncover<2->{Il utilise des données d'élévation, des élévations relatives, des pentes, direction de la pente et des bruits multi fractals pour sélectionner un des écosystèmes définis.}\newline

\uncover<3->{Les textures des végétations du sol sont générés à la volée, dépendant du niveau de détails et de l'écosystème.}\newline

\uncover<4->{L'écosystème détermine également le nombre de plantes par espèce, lesquelles sont placées aléatoirement.} \newline

\uncover<5->{La modélisation procédurale de la végétation produit des résultats fiables et sont déjà bien appliquées dans les jeux vidéos modernes, par exemple en utilisant le package commercial \textbf{SpeedTree}.}

\end{frame}
%-------------------------Frame 31: Les réseaux routiers----------------------------
\begin{frame}{Les réseaux routiers}

La génération de réseaux routiers pour les villes peut être faite par des méthodes variées, parmi lesquelles nous traitons les approches basées sur:
\begin{itemize}
	\item<2-> les pattern
	\item<3-> L-systems
	\item<4-> les simulations d'agents
	\item<5-> les champs tensoriels.\newline
\end{itemize}

\uncover<6->{La plus simple technique est de générer une \textbf{grille de carré dense} (comme dans Greuter et al. (2003)).}\newline
 
\uncover<7->{\textbf{Le bruit de déplacement} peut être ajouté aux points de la grille pour créer un \textbf{réseau moins répétitif}, mais le réalisme de cette technique est toujours limité.}

\end{frame}

%-------------------------Frame 32: Les réseaux routiers (2/9)---------------------------
\begin{frame}{Les réseaux routiers (2/9)}

Une méthode plus élaborée pour créer des routes est par le biais des modèles (ou "\textbf{templates}"), comme proposés par Sun et al. (2002).\newline

\uncover<2->{Ils observent plusieurs \textbf{patterns fréquents dans les réseaux routiers réels} et tentent de les reconstruire.}\newline

\uncover<3->{Pour chaque pattern, il y a un template correspondant:}
\begin{itemize}
	\item<4-> un template basé sur la population (implémenté comme un diagramme de Voronoi des centres de population)
	\item<5-> un raster et un template radial
	\item<6-> une template mixte. \newline
\end{itemize}

\uncover<7->{Les principales artères de la carte de la route sont les autoroutes, qui sont générées en premier en utilisant ces templates de pattern.}\newline

\uncover<8->{Des règles simples sont appliquées pour vérifier leur validité.}

\end{frame}

%-------------------------Frame 33: Les réseaux routiers (3/9)---------------------------
\begin{frame}{Les réseaux routiers (3/9)}

Des règles simples sont appliquées pour vérifier leur \textbf{validité}: \newline

\uncover<2->{Quand des zones d'impasse sont rencontrées (ex. les océans), elles sont annulées ou déviées.} \newline

\uncover<3->{Ensuite, les routes principales sont souvent courbées pour éviter les gradients d'élévation large.} \newline

\uncover<4->{Les régions qu'elles englobent sont complétées avec un raster de rues.}\newline

\end{frame}

%-------------------------Frame 34: Les réseaux routiers (4/9)--------------------------
\begin{frame}{Les réseaux routiers (4/9)}

Parish et Müller (2001) utilisent un \textbf{L-system étendu} pour faire croître leur réseau routier. \newline

\uncover<2->{Le L-system est conduit par des objectifs précis (\textbf{goal-driven}) et les objectifs en question sont la densité moyenne de la population (les rues tentent de connecter les centres de population et les patterns spécifiques de routes).}\newline

\uncover<3->{Des exemples de tels patterns sont le raster et le pattern radial.}\newline
 
\uncover<4->{Leur L-system est étendu avec des règles qui ont une tendance à connecter les routes nouvellement proposées avec les intersections existantes et des règles qui vérifient la validité de la route en respectant le \textbf{terrain impassable} et les \textbf{contraintes d'élévation}. Les rues sont aussi insérées dans les zones restantes comme des grilles simples.}

\end{frame}

%-------------------------Frame 35: Les réseaux routiers (5/9)--------------------------
\begin{frame}{Les réseaux routiers (5/9)}

Kelly et McCabe (2007) introduisent l'\textbf{éditeur de ville interactive CityGen}, dans lequel un utilisateur défini les routes principales en plaçant des noeuds dans le terrain 3D.\newline

\uncover<2->{Les régions inclues par ces routes peuvent être complétées avec une de ces trois patterns:}

\begin{itemize}
	\item<3-> les grilles style Manhattan
	\item<4-> les routes de croissance industrielle avec des routes sans issues
	\item<5-> les routes organiques comme par exemple les suburbaines nord-américaines.
\end{itemize}

\end{frame}

%-------------------------Frame 36: Les réseaux routiers (6/9)------------------------
\begin{frame}{Les réseaux routiers (6/9)}

Glass et al. (2006) décrivent plusieurs expérimentations de réplication de structures de route rencontrées dans les colonies informelles d'Afrique du Sud en utilisant la combinaison d'un \textbf{diagramme de Voronoi} pour les routes majeures avec des \textbf{L-systeme ou des subdivisions régulières avec ou sans bruit de déplacement} pour les routes mineures.\newline

\uncover<2->{Ils ont raisonnablement réussi à recréer les patterns observés.}

\end{frame}

%-------------------------Frame 37: Les réseaux routiers (7/9)------------------------
\begin{frame}{Les réseaux routiers (7/9)}

A la différence des approches basées sur les grammaires ou les patterns ci-dessus, Lechner et al. (2003) prennent une approche \textbf{basée sur l'agent}, dans laquelle ils divisent la ville en zones incluant non seulement les zones résidentielles, commerciales ou industrielles, mais également des zones spéciales comme les bâtiments gouvernementaux, les squares, et les institutions. \newline

\uncover<2->{Ils placent deux agents, appelé l'\textit{extender} et le \textit{connector}, sur une position de semence dans la carte du terrain.} \newline

\uncover<3->{L'extender recherche les zones non connectées dans la ville. Quand il trouve une telle zone qui est localisée non loin d'un réseau routier existant, il cherche le chemin le plus convenable pour connecter cette zone au réseau routier.} \newline

\uncover<4->{Dans Lechner et al. (2006), les auteurs étendent cette méthode avec, entre autres choses, des agents qui développent les petites rues.} \newline

\uncover<5->{Cette méthode donne des résultats plausibles, mais son désavantage est son \textbf{temps d'éxectution long}.}

\end{frame}

%-------------------------Frame 38: Les réseaux routiers (8/9)----------------------
\begin{frame}{Les réseaux routiers (8/9)}

Chen et al. (2008) proposent une modélisation interactive de réseaux routier en utilisant les \textbf{champs tensoriels. Ils définissent comment créer des patterns communs de routes (grille, radial, suivant les bords) en utilisant des champs tensoriels.} \newline

\uncover<3->{Un réseau routier est généré à partir d'un champ tensoriel, en traçant des lignes aérodynamiques à partir des \textbf{points de semence} dans les directions majeures des vecteurs propres jusqu'à ce qu'une condition d'arrêt soit remplie.} \newline

\uncover<4->{Ensuite, suivant la courbe tracée, de nouveaux points de semence sont placés pour tracer les lignes aérodynamiques dans la direction perpendiculaire (vecteur propre mineur).} \newline

\uncover<5->{Des utilisateurs peuvent placer des bases de champs tensoriels, comme un pattern radial, lisser le champ, ou utiliser un pinceau pour contraindre localement le champ vers une direction spécifique. Du bruit peut être appliqué pour rendre le réseau routier moins régulier et en conséquence plus plausible.}

\end{frame}

%-------------------------Frame 39: Les réseaux routiers (9/9)-------------------------
\begin{frame}{Les réseaux routiers (9/9)}

Dans les méthodes discutées, l'influence de la carte du terrain sous-jacente et du profil d'élévation est à degré variable prise en compte. \newline

\uncover<2->{La plupart des méthodes prennent seulement des \textbf{mesures basiques} pour éviter des routes montantes avec des \textit{pentes trop raides} et des \textit{routes qui traversent des corps d'eau}.} \newline

\uncover<3->{Kelly and McCabe (2007) planifient le chemin précis de leurs routes principales entre les noeuds mis en place par l'utilisateur pour avoir autant de possibilités d'élévation que possible.} \newline

\uncover<4->{Seulement, pour les terrains durs cette mesure ne sera pas adéquate et le terrain doit être modifié pour l'accomodation de la route. Bruneton and Neyret (2008) proposent une méthode simple et efficace pour mixer les profiles des routes dans les height-map en utilisant les \textit{shaders}.}

\end{frame}

%-------------------------Frame 40 : Les environnements urbains-----------------------
\begin{frame}{Les environnements urbains}

Kelly and McCabe (2006) donnent un point de vue élaboré de plusieurs approches pour la génération d'environnement urbains.\newline

\uncover<2->{Watson et al. (2008) donne un point de vue pratique de l'état de l'art.} \newline

\uncover<3->{L'approche la plus commune pour générer des villes procéduralement est de commencer par un réseau de routes dense et identifier les régions polygonales inclues dans les rues.}\newline

\uncover<4->{La subdivision de ces régions produisent des lots, pour lesquels différentes méthodes de subdivisions existent, voir par ex. Parish and Müller (2001) ou Kelly and McCabe (2007).}\newline

\uncover<5->{Pour remplir ces lots avec des immeubles, soit la forme du lot est utilisée directement comme empreinte d'un immeuble, ou l'empreinte d'un immeuble est adapté pour le lot.}\newline

\uncover<6->{En faisant simplement l'extrusion de l'empreinte jusqu'à une hauteur aléatoire, on peut générer une ville avec des gratte-ciels ou des immeubles de bureau.}
 
\end{frame}

%-------------------------Frame 41: Les environnements urbains (2/10)-------------------
\begin{frame}{Les environnements urbains (2/10)}

  Pour obtenir des formes d'immeubles intéressantes, plusieurs approches ont été mises en oeuvre.\newline
  
  \uncover<2->{Greuter et al. (2003) génère des immeubles de bureau en combinant plusieurs formes de primitives dans un plan d'étage et en faisant l'extrusion de cette forme pour différentes hauteurs.} \newline
  
  \uncover<3->{Parish and Müller (2001) commençent par un plan d'étage rectangulaire et appliquent un L-system pour affiner l'immeuble. }\newline
  
  \uncover<4->{Ces approches sont surtout utiles pour des modèles simples d'immeubles de bureau.}\newline
  
 \uncover<5->{Wonka et al. (2003) introduisent le concept de \textit{split grammar}, une grammaire formelle indépendante du contexte (context-free formal grammar) dont le design est fait pour produire des modèles d'immeubles.}

\end{frame}

%-------------------------Frame 42: Les environnements urbains (3/10)-------------------
\begin{frame}{Les environnements urbains (3/10)}

Le split grammar ressemble à un L-system, mais il est basé sur des formes comme éléments primitifs à la place des symboles. \newline
 
 \uncover<2->{Dans leur système, un style spécific d'immeuble peut être acquise en prennant un attribut au début du symbole, lequel est propagé pendant la réécriture.} \newline
 
 \uncover<3->{Dans un modèle d'immeuble, le style peut être différent pour chaque étage (ex. un immeuble d'appartement avec des boutiques au le rez-de-chaussé).} \newline
 
 \uncover<4->{Leur approche repose surtout sur la génération cohérente et plausible des façades pour des immeubles au forme relativement simple. }

\end{frame}

%-------------------------Frame 43: Les environnements urbains (4/10)-------------------
\begin{frame}{Les environnements urbains (4/10)}

 Larive and Gaildrat (2006) utilise un grammaire de la même sorte, appelé \textit{wall grammar}. Avec cette grammaire, ils sont capables de générer des murs d'immeuble avec des détails géométriques additionnels, comme les balcons.\newline

\uncover<2->{Müller et al. (2006) appliquent une autre type de grammaire, appelée \textit{shape grammar}.} \newline

\uncover<3->{La propriété principale d'un shape grammar est qu'il utilise des règles dépendant du contexte (context-sensitive), tandis qu'une split grammar utilisent des règles indépendantes du contexte (context-free), qui dans ce cas permet la possibilité de modéliser les toits et les formes arrondies (rotated shapes).} \newline

\uncover<4->{Ils commencent avec une réunion de plusieurs formes volumétriques lesquelles définissent les bords des immeubles.} 

\end{frame}

%-------------------------Frame 44: Les environnements urbains (5/10)--------------------
\begin{frame}{Les environnements urbains (5/10)}


Cette forme est ensuite divisée en étages et les façades produites sont subdivisées en murs, fenêtres, et portes par le moyen d'un système de grammaire. \newline

\uncover<2->{Dans une étape finale, le toit est construit sur le sommet de l'immeuble.
Fig 1f montre un réseau routier et Fig 1g la ville correspondante générée par leur produit commercial, CityEngine (Procedural, inc. 2009).}\newline

\uncover<3->{En plus des immeubles d'affaire bien connus, la grammaire peut aussi modéliser des immeubles résidentiels, ex. les maisons suburbaines ou les anciennes villas romaines.}\newline

\end{frame}

%-------------------------Frame 45: Les environnements urbains (6/10)--------------------
\begin{frame}{Les environnements urbains (6/10)}

Bien que les shape grammars dans Müller et al. (2006) peuvent générer des modèles d'immeubles visuellement convainquants, Finkenzeller and Bender (2008) notent que l'information sémantique à propos du rôle de chaque forme dans l'immeuble final est manquante.\newline

\uncover<2->{Ils proposent de capturer cette information sémantique dans un graphe typé. Leur workflow comprend trois étapes.}

\begin{itemize}
\item<3-> En commençant par des traits de contour brute, un graphe de style d'immeuble peut être appliqué à ce modèle.
\item<4-> Cela produit une représentation intermédiaire de graphe sémantique de l'immeuble, laquelle peut être modifiée ou regénérée avec un style différent.
\item<5-> Dans la dernière étape, la géométrie est crée sur la base du modèle intermédiaire, et les textures sont appliquées, produisant l'immeuble 3D final.
\end{itemize}
 
\end{frame}

%-------------------------Frame 46: Les environnements urbains (7/10)-------------------
\begin{frame}{Les environnements urbains (7/10)}

Finkenzeller (2008) décrit avec plus de détails la génération des façades et des toits de leur système (voir fig. 1c).\newline

\uncover<2->{Young et al. (2004) décrivent une méthode pour créer des maisons au style vernaculaire du sud-est de la Chine en utilisant un shape grammar étendue.} \newline

\uncover<3->{La grammaire est hierarchique et commencent au niveau de la ville (tandis que dans d'autres méthodes un shape grammar est appliqué à l'empreinte d'un immeuble individuel).}\newline

\uncover<4->{La grammaire produit ensuite les rues, les blocs d'habitation, les routes, et même les maisons de productions avec des composantes comme les portails, les fenêtres, les murs et les toîts.}

\end{frame}

%-------------------------Frame 47: Les environnements urbains (8/10)-------------------
\begin{frame}{Les environnements urbains (8/10)}
 
A travers des règles de contrôles (définissant par exemple, les contraintes de ratio de composants) la validité des immeubles peut être testée.\newline
 
\uncover<2->{En appliquant ce système de grammaire, une ville ancienne typique du sud-est de la Chine peut être généré avec des résultats plausibles, puisque les style d'immeuble de ces villes est très rigidement structuré.}\newline

\uncover<3->{Müller et al. (2007) utilisent une approche très différente pour la construction des façades des immeubles.}\newline

\uncover<4->{Leur méthode prend un image unique de la façade de l'immeuble réel comme entrée et est capable de reconstruire un modèle détaillé de la façade en 3D, en utilisant une combinaison d'image et de génération de shape grammar.}\newline

\end{frame}

%-----------------------Frame 48: Les environnements urbains (9/10)---------------------
\begin{frame}{Les environnements urbains (9/10)}

Bien que les méthodes ci-dessus donnent rapidement et visuellement des résultats plaisants, les villes qu'elles génèrent manque de structure réaliste.\newline 

\uncover<2->{Une nouvelle recherche incorpore les théories et les modèles d'aménagement de territoire urbain existantes dans le processus de génération.} \newline

\uncover<3->{Groenewegen et al. (2009) présentent une méthode qui génère une distribution de différents types de districts selon les modèles d'aménagement urbain dans l'Europe de l'Ouest et l'Amérique du Nord.}\newline

\uncover<4->{Ils prennent en compte un grand éventail de facteurs significatifs, incluant le plan historique de la ville et de l'attraction que certains types de terrain (côteaus, océans, rivières) peuvent avoir par ex. les districs industriels ou résidentiels de grande classe. }



\end{frame}
%-------------------------Frame 49: Les environnements urbains (10/10)-------------
\begin{frame}{Les environnements urbains (10/10)}


Weber et al. (2009) utilisent des modèles comparables (comparaison un peu simplifiée) pour une simulation d'une ville dans le temps. \newline

\uncover<2->{Leur système est rapide (environ 1s pour un an de simulation) et interactif, ce qui veut dire que l'utilisateur peut guider la simulation en changeant les routes ou en mettant une peinture pour le sol utilisé sur le terrain.}

\end{frame}
%-------------------------Frame : ----------------------------
\begin{frame}{}

\uncover<3->{}
\begin{uncoverenv}<4->
\end{uncoverenv}


\begin{itemize}
	\item
	\item
\end{itemize}
\end{frame}
%-------------------------Frame : ----------------------------
\begin{frame}{}

\end{frame}
%-------------------------Frame : ----------------------------
\begin{frame}{}

\end{frame}
%-------------------------Frame : ----------------------------
\begin{frame}{}

\end{frame}
%-------------------------Frame : ----------------------------
\begin{frame}{}

\end{frame}
%-------------------------Frame : ----------------------------
\begin{frame}{}

\end{frame}
%-------------------------Frame : ----------------------------
\begin{frame}{}

\end{frame}
%-------------------------Frame : ----------------------------
\begin{frame}{}

\end{frame}
%-------------------------Frame : ----------------------------
\begin{frame}{}

\end{frame}
%-------------------------Frame : ----------------------------
\begin{frame}{}

\end{frame}
%-------------------------Frame : ----------------------------
\begin{frame}{}

\end{frame}
%-------------------------Frame : ----------------------------
\begin{frame}{}

\end{frame}
%-------------------------Frame : ----------------------------
\begin{frame}{}

\end{frame}
%-------------------------Frame : ----------------------------
\begin{frame}{}

\end{frame}
%-------------------------Frame : ----------------------------
\begin{frame}{}

\end{frame}
%-------------------------Frame : ----------------------------
\begin{frame}{}

\end{frame}
%-------------------------Frame : ----------------------------
\begin{frame}{}

\end{frame}
%-------------------------Frame : ----------------------------
\begin{frame}{}

\end{frame}
%-------------------------Frame : ----------------------------
\begin{frame}{}

\end{frame}
%-------------------------Frame : ----------------------------
\begin{frame}{}

\end{frame}
%-------------------------Frame : ----------------------------
\begin{frame}{}

\end{frame}
%-------------------------Frame : ----------------------------
\begin{frame}{}

\end{frame}
%-------------------------Frame : ----------------------------
\begin{frame}{}

\end{frame}
%-------------------------Frame : ----------------------------
\begin{frame}{}

\end{frame}
%-------------------------Frame : ----------------------------
\begin{frame}{}

\end{frame}
%-------------------------Frame : ----------------------------
\begin{frame}{}

\end{frame}
%-------------------------Frame : ----------------------------
\begin{frame}{}

\end{frame}
%-------------------------Frame : ----------------------------
\begin{frame}{}

\end{frame}
%-------------------------Frame : ----------------------------
\begin{frame}{}

\end{frame}
%-------------------------Frame : ----------------------------
\begin{frame}{}

\end{frame}
%-------------------------Frame : ----------------------------
\begin{frame}{}

\end{frame}
%-------------------------Frame : ----------------------------
\begin{frame}{}

\end{frame}
%-------------------------Frame : ----------------------------
\begin{frame}{}

\end{frame}
%-------------------------Frame : ----------------------------
\begin{frame}{}

\end{frame}
%-------------------------Frame : ----------------------------
\begin{frame}{}

\end{frame}
%-------------------------Frame : ----------------------------
\begin{frame}{}

\end{frame}
%-------------------------Frame : ----------------------------
\begin{frame}{}

\end{frame}
%-------------------------Frame : ----------------------------
\begin{frame}{}

\end{frame}
%-------------------------Frame : ----------------------------
\begin{frame}{}

\end{frame}
%-------------------------Frame : ----------------------------
\begin{frame}{}

\end{frame}
%-------------------------Frame : ----------------------------
\begin{frame}{}

\end{frame}
%-------------------------Frame : ----------------------------
\begin{frame}{}

\end{frame}
%-------------------------Frame : ----------------------------
\begin{frame}{}

\end{frame}
%-------------------------Frame : ----------------------------
\begin{frame}{}

\end{frame}
%-------------------------Frame : ----------------------------
\begin{frame}{}

\end{frame}
%-------------------------Frame : ----------------------------
\begin{frame}{}

\end{frame}
%-------------------------Frame : ----------------------------
\begin{frame}{}

\end{frame}
%-------------------------Frame : ----------------------------
\begin{frame}{}

\end{frame}
%-------------------------Frame : ----------------------------
\begin{frame}{}

\end{frame}
%-------------------------Frame : ----------------------------
\begin{frame}{}

\end{frame}
%-------------------------Frame : ----------------------------
\begin{frame}{}

\end{frame}
%-------------------------Frame : ----------------------------
\begin{frame}{}

\end{frame}
%-------------------------Frame : ----------------------------
\begin{frame}{}

\end{frame}
%-------------------------Frame : ----------------------------
\begin{frame}{}

\end{frame}
%-------------------------Frame : ----------------------------
\begin{frame}{}

\end{frame}
%-------------------------Frame : ----------------------------
\begin{frame}{}

\end{frame}
%-------------------------Frame : ----------------------------
\begin{frame}{}

\end{frame}
%-------------------------Frame : ----------------------------
\begin{frame}{}

\end{frame}
%-------------------------Frame : ----------------------------
\begin{frame}{}

\end{frame}
%-------------------------Frame : ----------------------------
\begin{frame}{}

\end{frame}
%-------------------------Frame : ----------------------------
\begin{frame}{}

\end{frame}
%-------------------------Frame : ----------------------------
\begin{frame}{}

\end{frame}
%-------------------------Frame : ----------------------------
\begin{frame}{}

\end{frame}
%-------------------------Frame : ----------------------------
\begin{frame}{}

\end{frame}
%-------------------------Frame : ----------------------------
\begin{frame}{}

\end{frame}
%-------------------------Frame : ----------------------------
\begin{frame}{}

\end{frame}
%-------------------------Frame : ----------------------------
\begin{frame}{}

\end{frame}
%-------------------------Frame : ----------------------------
\begin{frame}{}

\end{frame}
%-------------------------Frame : ----------------------------
\begin{frame}{}

\end{frame}
%-------------------------Frame : ----------------------------
\begin{frame}{}

\end{frame}
%-------------------------Frame : ----------------------------
\begin{frame}{}

\end{frame}
%-------------------------Frame : ----------------------------
\begin{frame}{}

\end{frame}
%-------------------------Frame : ----------------------------
\begin{frame}{}

\end{frame}
%-------------------------Frame : ----------------------------
\begin{frame}{}

\end{frame}
%-------------------------Frame : ----------------------------
\begin{frame}{}

\end{frame}
%-------------------------Frame : ----------------------------
\begin{frame}{}

\end{frame}
%-------------------------Frame : ----------------------------
\begin{frame}{}

\end{frame}
%-------------------------Frame : ----------------------------
\begin{frame}{}

\end{frame}
%-------------------------Frame : ----------------------------
\begin{frame}{}

\end{frame}
%-------------------------Frame : ----------------------------
\begin{frame}{}

\end{frame}
%-------------------------Frame : ----------------------------
\begin{frame}{}

\end{frame}
%-------------------------Frame : ----------------------------
\begin{frame}{}

\end{frame}
%-------------------------Frame : ----------------------------
\begin{frame}{}

\end{frame}
%-------------------------Frame : ----------------------------
\begin{frame}{}

\end{frame}
%-------------------------Frame : ----------------------------
\begin{frame}{}

\end{frame}
%-------------------------Frame : ----------------------------
\begin{frame}{}

\end{frame}
%-------------------------Frame : ----------------------------
\begin{frame}{}

\end{frame}
%-------------------------Frame : ----------------------------
\begin{frame}{}

\end{frame}
%-------------------------Frame : ----------------------------
\begin{frame}{}

\end{frame}
%-------------------------Frame : ----------------------------
\begin{frame}{}

\end{frame}
%-------------------------Frame : ----------------------------
\begin{frame}{}

\end{frame}
%-------------------------Frame : ----------------------------
\begin{frame}{}

\end{frame}
%-------------------------Frame : ----------------------------
\begin{frame}{}

\end{frame}
%-------------------------Frame : ----------------------------
\begin{frame}{}

\end{frame}
%-------------------------Frame : ----------------------------
\begin{frame}{}

\end{frame}
%-------------------------Frame : ----------------------------
\begin{frame}{}

\end{frame}
%-------------------------Frame : ----------------------------
\begin{frame}{}

\end{frame}
%-------------------------Frame : ----------------------------
\begin{frame}{}

\end{frame}
%-------------------------Frame : ----------------------------
\begin{frame}{}

\end{frame}
%-------------------------Frame : ----------------------------
\begin{frame}{}

\end{frame}
%-------------------------Frame : ----------------------------
\begin{frame}{}

\end{frame}
%-------------------------Frame : ----------------------------
\begin{frame}{}

\end{frame}
%-------------------------Frame : ----------------------------
\begin{frame}{}

\end{frame}
%-------------------------Frame : ----------------------------
\begin{frame}{}

\end{frame}
%-------------------------Frame : ----------------------------
\begin{frame}{}

\end{frame}
%-------------------------Frame : ----------------------------
\begin{frame}{}

\end{frame}
%-------------------------Frame : ----------------------------
\begin{frame}{}

\end{frame}
%-------------------------Frame : ----------------------------
\begin{frame}{}

\end{frame}
%-------------------------Frame : ----------------------------
\begin{frame}{}

\end{frame}
%-------------------------Frame : ----------------------------
\begin{frame}{}

\end{frame}
%-------------------------Frame : ----------------------------
\begin{frame}{}

\end{frame}
%-------------------------Frame : ----------------------------
\begin{frame}{}

\end{frame}
%-------------------------Frame : ----------------------------
\begin{frame}{}

\end{frame}
%-------------------------Frame : ----------------------------
\begin{frame}{}

\end{frame}
%-------------------------Frame : ----------------------------
\begin{frame}{}

\end{frame}
%-------------------------Frame : ----------------------------
\begin{frame}{}

\end{frame}
%-------------------------Frame : ----------------------------
\begin{frame}{}

\end{frame}
%-------------------------Frame : ----------------------------
\begin{frame}{}

\end{frame}
%-------------------------Frame : ----------------------------
\begin{frame}{}

\end{frame}
%-------------------------Frame : ----------------------------
\begin{frame}{}

\end{frame}
%-------------------------Frame : ----------------------------
\begin{frame}{}

\end{frame}
%-------------------------Frame : ----------------------------
\begin{frame}{}

\end{frame}
%-------------------------Frame : ----------------------------
\begin{frame}{}

\end{frame}
%-------------------------Frame : ----------------------------
\begin{frame}{}

\end{frame}
%-------------------------Frame : ----------------------------
\begin{frame}{}

\end{frame}
%-------------------------Frame : ----------------------------
\begin{frame}{}

\end{frame}
%-------------------------Frame : ----------------------------
\begin{frame}{}

\end{frame}
%-------------------------Frame : ----------------------------
\begin{frame}{}

\end{frame}
%-------------------------Frame : ----------------------------
\begin{frame}{}

\end{frame}
%-------------------------Frame : ----------------------------
\begin{frame}{}

\end{frame}
%-------------------------Frame : ----------------------------
\begin{frame}{}

\end{frame}
%-------------------------Frame : ----------------------------
\begin{frame}{}

\end{frame}
%-------------------------Frame : ----------------------------
\begin{frame}{}

\end{frame}
%-------------------------Frame : ----------------------------
\begin{frame}{}

\end{frame}
%-------------------------Frame : ----------------------------
\begin{frame}{}

\end{frame}
%-------------------------Frame : ----------------------------
\begin{frame}{}

\end{frame}
%-------------------------Frame : ----------------------------
\begin{frame}{}

\end{frame}
%-------------------------Frame : ----------------------------
\begin{frame}{}

\end{frame}
%-------------------------Frame : ----------------------------
\begin{frame}{}

\end{frame}
%-------------------------Frame : ----------------------------
\begin{frame}{}

\end{frame}
%-------------------------Frame : ----------------------------
\begin{frame}{}

\end{frame}
%-------------------------Frame : ----------------------------
\begin{frame}{}

\end{frame}
%-------------------------Frame : ----------------------------
\begin{frame}{}

\end{frame}
%-------------------------Frame : ----------------------------
\begin{frame}{}

\end{frame}
%-------------------------Frame : ----------------------------
\begin{frame}{}

\end{frame}
%-------------------------Frame : ----------------------------
\begin{frame}{}

\end{frame}
%-------------------------Frame : ----------------------------
\begin{frame}{}

\end{frame}
%-------------------------Frame : ----------------------------
\begin{frame}{}

\end{frame}
%-------------------------Frame : ----------------------------
\begin{frame}{}

\end{frame}
%-------------------------Frame : ----------------------------
\begin{frame}{}

\end{frame}
%-------------------------Frame : ----------------------------
\begin{frame}{}

\end{frame}

\end{document}
